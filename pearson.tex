A Pearson density p is defined to be any valid solution to the differential equation (cf. Pearson 1895, p. 381)

{\displaystyle {\frac {p'(x)}{p(x)))+{\frac {a+(x-\lambda )}{b_{0}+b_{1}(x-\lambda )+b_{2}(x-\lambda )^{2))}=0.\qquad (1)}
with:

{\displaystyle {\begin{aligned}b_{0}&={\frac {4\beta _{2}-3\beta _{1)){10\beta _{2}-12\beta _{1}-18))\mu _{2},\\a=b_{1}&={\sqrt {\mu _{2))}{\sqrt {\beta _{1))}{\frac {\beta _{2}+3}{10\beta _{2}-12\beta _{1}-18)),\\b_{2}&={\frac {2\beta _{2}-3\beta _{1}-6}{10\beta _{2}-12\beta _{1}-18)).\end{aligned))}
According to Ord,[3] Pearson devised the underlying form of Equation (1) on the basis of, firstly, the formula for the derivative of the logarithm of the density function of the normal distribution (which gives a linear function) and, secondly, from a recurrence relation for values in the probability mass function of the hypergeometric distribution (which yields the linear-divided-by-quadratic structure).

In Equation (1), the parameter a determines a stationary point, and hence under some conditions a mode of the distribution, since

{\displaystyle p'(\lambda -a)=0}
  follows directly from the differential equation.
  
  Since we are confronted with a first order linear differential equation with variable coefficients, its solution is straightforward:
    
  {\displaystyle p(x)\propto \exp \left(-\int {\frac {x-a}{b_{2}x^{2}+b_{1}x+b_{0))}\,dx\right).}
    The integral in this solution simplifies considerably when certain special cases of the integrand are considered. Pearson (1895, p. 367) distinguished two main cases, determined by the sign of the discriminant (and hence the number of real roots) of the quadratic function
    
    {\displaystyle f(x)=b_{2}x^{2}+b_{1}x+b_{0}.\qquad (2)}
